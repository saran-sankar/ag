\documentclass[ag.tex]{subfiles}

\begin{document}

\chapter{Schemes}

The central objects of our study are schemes,  which are mathematical structures that generalize the notion of algebraic varieties in a similar manner to how manifolds generalize subsets of Euclidean spaces.  An algebraic variety is often defined as a set of solutions of a system of polynomials over some field, which gives an algebraic description of geometric objects such as curves and surfaces.  It may be noted that varieties and appropriately defined maps between them form a category.

\section{From varieties to spectra}

We may briefly introduce affine algebraic varieties, a type of variety,  in order to motivate schemes.

\begin{definition}
An \textit{affine algebraic set} $X$ is the set of solutions in an algebraically closed field $k$ of a system of polynomials in $k[x_1,x_2,...,x_n]$ for some $n$.
\end{definition}

\begin{definition}
An \textit{affine (algebraic) variety} $X$ is an affine algebraic set which is not the union of two proper affine algebraic subsets.
\end{definition}

The important observation is that we can associate a "special" ring to any affine variety which reflects all properties of the variety.  This is the quotient of the polynomial ring $k[x_1,x_2,...,x_n]$ by the ideal $I$ of polynomials that are zero on $X$:

\begin{definition}
Let $X$ be an affine variety and $I$ be the ideal of $k[x_1,x_2,...,x_n]$ generated by all polynomials that are zero on $X$. Then we call the quotient ring $k[X] := k[x_1,x_2,...,x_n] / I$ the \textit{coordinate ring} of $X$.
\end{definition}

The idea is to go in the other direction i.e., to start with an arbitrary ring and associate a geometric object to it. This program will yield the notion of schemes, which is much more general than that of varieties. 

A way to recover an affine variety $X$ from its coordinate ring $k[X]$ is to use that fact that there is a one-to-one correspondence between "points" $x$ of $X$ and maximal ideals $m_x$ of $k[X]$.  Therefore a first candidate for the geometric object associated with an arbitrary ring $A$ is the set of all maximal ideals of $A$, which we shall denote by $m$-Spec$A$.  But this assignment fails to obey functoriality for the following reason.

Suppose $f: A \to B$ is a ring homomorphism.  For our assignment to be functorial we must find a morphism between $m$-Spec$A$ and $m$-Spec$B$. But there is no natural way to do this since for maximal ideals $a \subset A$ and $b \subset B$, neither $f^{-1} (b)$ or $f(a)$ is necessarily maximal. 

If we instead consider the set of all prime ideals of $A$, also called the \textit{spectrum} of $A$, Spec $A \supset$ $m$-Spec $A$, then this difficulty disappears owing to Proposition \ref{primes_map_to_primes}.  In other words, we have a contravariant functor from the category of (commutative) rings to the category of spectra (or the category of topological spaces,  as we shall see).  We derive a justification for the claim that Spec$A$ is "geometrical" by successfully assigning it a topology.

\begin{definition} 
The \textit{Zariski topology} of Spec $A$ is given by setting the collection of closed sets of Spec $A$ as
\begin{equation*}
\text{\{$V_I$: $I$ is an ideal of $A$\}}
\end{equation*}
where $V_I$ is the set of all prime ideals containing $I$.
\end{definition} 

It can be checked that this indeed defines a topology.

\section{Sheaves}

Along with Spec $A$,  the building blocks of a scheme is given by sheaves.  We first define a weaker notion, namely that of a presheaf.

\begin{definition}
Let $X$ be a topological space and $Open(X)$ be the category in which the objects are open sets of $X$ and morphisms are inclusion maps. A \textit{presheaf} is a contravariant functor

\begin{equation*}
\mathcal{F}: Open(X) \to \mathcal{C}
\end{equation*}
where $\mathcal{C}$ is some category.
\end{definition}

The maps $\mathcal{F}(V) \to \mathcal{F}(U)$ where $U \subset V$ are open sets of $X$ is denoted by $\rho_U^V$.  If $\mathcal{C}$ is the category of groups, rings or modules then we say $\mathcal{F}$ is a presheaf of groups, rings or modules respectively. 

\begin{exmp}\label{sheaf_of_rings_of_cont_fun}
Presheaf of the rings of $C^0$ functions defined on open sets of $X$. Here the homomorphisms $\rho_U^V$ give restrictions of functions $f: V \to \mathbb{R}$ to $U$, i.e.  $\rho_U^V (f) = f|_{U}$.
\end{exmp}

The most important presheaf for us is the structure presheaf on Spec $A$:

\begin{definition}
The \textit{structure presheaf} of Spec $A$, denoted by $\mathcal{O}$,  is the contravariant functor given by 

\begin{equation}
\mathcal{O}(U) = \varprojlim \mathcal{O}(D(f))
\end{equation}
and 

\begin{equation}
\rho_U^V(\{v_\alpha\}) = \{v_\beta\}
\end{equation}
where 
\end{definition}

It can be verified that $\mathcal{O}$ is a functor as claimed.  

\begin{definition}
Suppose the category $\mathcal{C}$ has a terminal object $T$ (see Definition \ref{terminal object}).  Then a $\textit{section}$ of $\mathcal{F}(U)$ is a morphism $u: T \to \mathcal{F}(U)$.
\end{definition}

Note that if $\mathcal{F}(U)$ is a set then $u$ can be understood as "picking out" an element (say $t_u$) in $\mathcal{F}(U)$.  Thus for any morphism $v$ from $\mathcal{F}(U)$ to some object $C$ in $\mathcal{C}$,  we can regard the composition $v \circ u$ as a generalization of $v(t_u)$. With this given, we define a sheaf as a special case of presheaves:

\begin{definition}
A \textit{sheaf} is a presheaf that satisfies the following conditions
\begin{enumerate}
\item 
\end{enumerate}
\end{definition}

\begin{exmp}
The presheaf in Example \ref{sheaf_of_rings_of_cont_fun} and the structure presheaf on Spec $A$ are sheaves.
\end{exmp}

\section{Schemes}

First we introduce the category of ringed spaces.

\begin{definition}
A \textit{ringed space} is a pair $(X, \mathcal{O})$ where $X$ is a topological space and $\mathcal{O}$ is a sheaf of rings. A \textit{morphism of ringed spaces} $\varphi: (X, \mathcal{O}_X) \to (Y, \mathcal{O}_Y)$ is a map $\varphi: X \to Y$ along with a natural transformation $\psi$ from $\mathcal{O}_Y$ to $\mathcal{O}_X \circ \varphi^{-1}$.
\end{definition}

Note that naturalness means $\psi$ is a collection of homomorphisms $\psi_U: \mathcal{O}_Y (U) \to \mathcal{O}_X (\varphi^{-1}(U))$ for open sets $U \subset Y$ such that the following diagram commutes for all open sets $U \subset V$ \\

It can be checked that for any ring $A$, (Spec $A$,  $\mathcal{O}_A$) where $\mathcal{O}_A$ is the structure sheaf on Spec $A$ is a ringed space.  We denote this simply as Spec $A$ from now on. Finally we define a scheme:

\begin{definition}
A \textit{scheme} is a ringed space $(X, \mathcal{O})$ for which every point has a neighborhood $U$ such that the ringed space $(U, \mathcal{O}|_{U})$ is isomorphic to Spec $A$ for some ring $A$. 
\end{definition}

\end{document}